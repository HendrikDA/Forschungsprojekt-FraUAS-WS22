\chapter{Problem Statement}
\label{ch:problem}

Creating individual profiles of users and utilizing these to display personalized ads have lead to valuable business models. Creating laws and policies that make doing so more difficult can have shattering effects on a business's revenue.

The topic that this research paper analyzes is possible alternative methods to cookies to track users' online behavior and collect their data. This will be done within the realm of Web3, which is built upon technologies such as blockchain and cryptocurrency wallets. Using Web3 technologies, it might be possible to replace, or rather supplement, cookies and continue to allow for detailed tracking of users. Even in an online world with more strict privacy policies and regulations where third-party cookies are becoming less frequent.

The research question at hand is thus \textit{How do NFTs and cryptocurrency wallets work as an alternative method to track users and collect their data? What are the pros and cons of using these technologies to do so?}. This will be answered by giving an overview of the necessary technologies in chapter \ref{ch:bg_tech} and what current literature and approaches exist on the subject in chapter \ref{ch:SOTA}. Chapter \ref{ch:methodology} will then go over the methodology to analyze how well Web3 technologies might supplement cookies. This is done based on a case study of online stores, where a user connects their wallet to the website. These results are then discussed in chapter \ref{ch:results}.