\chapter{Results and Discussion}
\label{ch:results}


%
% Section: Results
%
\section{Results}
\label{sec:results:results}
Maybe include a table with a comparison of some key features for tracking user data. Cookies and NFTs side by side.


%
% Section: Research Challenges
%
\section{Research Challenges}
\label{sec:results:researchChallenges}
Todo: Go into depth on what challenges you faced while researching. I.e. that there is a research gap for gathering user data with wallets and NFTs

%
% Section: Problems
%
\section{Problems}
\label{sec:results:problems}

% Quality of Data
\subsection{Quality of Data}
\label{sub:results:data}
Todo: How is the data that is gathered different from cookies?



% Sharing Your Wallet's Content
\subsection{Privacy - Sharing Your Wallet's Content}
\label{sec:results:walletContent}
It might be a problem who an individual shows their wallet to, i.e. with which platforms a user connects their wallet to. Because an online store receives the public address of a user, and can thus see the contents of their wallet, certain privacy issues may arise with SSO.

MetaMask even warns to be careful about which websites access is granted to. It recommends to check how well-known a project is, how often the user intends on using the dApp, and whether or not known security breaches have happened before on the connecting website. \cite{metaMask}

Although a user's name is not directly associated to their wallet, it is often possible to obtain this information by analyzing the wallet's activity and transaction history. Aggregating this information, it can be possible to trace back who the wallet belongs to.

This may force users to own several wallets. Each time a user connects to an online store, they would have to think about which wallet they want to use, meaning which wallet's content they want to share with the website they are connecting to and whether or not they want information from this site written to their address. Again, once a transaction takes place on the blockchain, it is immutable and cannot be reversed. That bit of information is thus permanently and publicly recorded on the blockchain and associated with the given public address of the user.


%
% Section: High Entry Barrier
%
\subsection{High Entry Barrier}
\label{sub:results:barrier}
Todo: How is the high entry barrier of NFTs going to influence the wide spread use of tracking via NFTs? It might not be very practical to track users this way.

%
% Section: High Gas Prices
%
\subsection{High Gas Prices}
\label{sub:results:barrier}
Todo: How is the high entry barrier of NFTs going to influence the wide spread use of tracking via NFTs? It might not be very practical to track users this way.


%
% Section: Application Potential
%
\section{Application Potential}
\label{sec:results:potential}
Go into detail of what potential use cases may arise from NFTs being used in this way. For example, if I buy a Werder Ticket and that gets saved into my wallet, and I then get a hotel somewhere in Bremen, they may see that I'm there for the game and may give me a discount or something. The prerequisite for something like that is that everyone supports NFTs.



%
% Section: Discussion
%
\section{Discussion}
\label{sec:results:discussion}
