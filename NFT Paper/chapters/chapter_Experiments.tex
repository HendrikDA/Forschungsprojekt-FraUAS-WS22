\chapter{Results and Discussion}
\label{ch:results}


%
% Section: Results
%
\section{Results}
\label{sec:results:results}
This chapter discusses the results of this paper by covering the downsides of using this method to track user data.


%
% Section: Research Challenges
%
\section{Research Challenges}
\label{sec:results:researchChallenges}
The lack of existing research around this subject comes with both advantages and disadvantages. A gap in research presents a large opportunity to have an impact in the research field at hand. Analyzing problems that no one has done before and being the first to market with research has its perks. On the other hand, not having any existing literature or research to use as a foundation of this paper deemed to be a large challenge. With everything being unanalyzed, in what direction should the research go? Narrowing this down and gathering a solid foundation to base this research on was a challenge in and of itself.


% Quality of Data
\subsection{Comparison of Tracking Methods}
\label{sub:results:data}
Todo: Add a table with comparisons. Pros and cons of each.



% Sharing Your Wallet's Content
\subsection{Privacy - Sharing Your Wallet's Content}
\label{sec:results:walletContent}
It might be a problem who an individual shows their wallet to, i.e. with which platforms a user connects their wallet to. Because an online store receives the public address of a user, and can thus see the contents of their wallet, certain privacy issues may arise with SSO.

MetaMask even warns to be careful about which websites access is granted to. It recommends to check how well-known a project is, how often the user intends on using the dApp, and whether or not known security breaches have happened before on the connecting website. \cite{metaMask}

Although a user's name is not directly associated to their wallet, it is often possible to obtain this information by analyzing the wallet's activity and transaction history. Aggregating this information, it can be possible to trace back who the wallet belongs to.

This may force users to own several wallets. Each time a user connects to an online store, they would have to think about which wallet they want to use, meaning which wallet's content they want to share with the website they are connecting to and whether or not they want information from this site written to their address. Again, once a transaction takes place on the blockchain, it is immutable and cannot be reversed. That bit of information is thus permanently and publicly recorded on the blockchain and associated with the given public address of the user.


%
% Section: High Entry Barrier
%
\subsection{High Entry Barrier}
\label{sub:results:barrier}
The technical possibilities of NFTs and Wallets are seemingly endless. However, actually creating a wide user-base, where it would be worthwhile to gift NFTs with real-world purchases and track users with the data of wallets would be a large challenge.

One problematic factor is that the technical entry barrier to Web3 is very high, compared to previous iterations of the web. In Web2 a user virtually only requires a smartphone and an internet connection to interact with social media sites. Comparing this to Web3, a user is required to have at least a very basic understanding of blockchain, which is a complicated technology. On top of this, they are required to create a wallet and understand how to use it and how it differs from conventional accounts created in the space of Web2.

This is a lot to ask of non-technical users. Getting Wallets and NFTs to the masses will require the products leveraging this technology to mature. They must become more user friendly and more seamlessly integrated into websites. Otherwise, the technical understanding will be too high and only a niche set of users will adopt the technology in their every day lives.

%
% Section: High Gas Prices
%
\subsection{High Gas Prices}
\label{sub:results:barrier}
According to the official Ethereum page, gas prices in regard to cryptocurrencies is the required computational effort in order to interact with the Ethereum network (blockchain). \cite{ether}

Because the blockchain consists of highly complex cryptographic calculations, it is not a trivial matter to execute transactions on it. Even though the blockchain is decentralized, it is still required by some computer to execute the computation. The gas prices for the example in section \ref{sec:methodology:code} totalled a network fee of 11.91€. A higher price than the NFT used in the example itself, which cost 7.27€.

Discussing gas prices in-depth is worthy of its own research. In short, the high gas prices for interaction with the blockchain may be another hindering factor for wide-spread adoption. Although it is technically trivial to send a user an NFT after purchasing a real-world good, it might come with too high costs. These costs would need to be reduced before wide-spread adoption is possible.

