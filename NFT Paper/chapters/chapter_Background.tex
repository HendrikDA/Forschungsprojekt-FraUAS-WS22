\chapter{Background Information}
\label{ch:background}
In order to discuss possible ways to track user data using NFTs, it is important to gain an overview of the required technologies behind NFTs. These technologies are blockchain, smart contracts, NFTs themselves and cryptocurrency wallets. After these technologies, the function of cookies and how they are implemented in order to track user data is briefly discussed.


% Overview of what blockchain is
% What are NFTs?
% Smart contracts
% What are Wallets?
% What are cookies?
% Privacy and policies
% What kind of user data is tracked online?

% \cite{bitcoin} \cite{blockchain}

%
% Section: Overview of Blockchain
%
\section{Blockchain}
\label{sec:background:blockchain}
Blockchain technology allows for peer-to-peer electronic cash payments. What makes blockchain different from other forms of electronic payments is that it takes out the trusted third party acting as a middleman between each transaction \cite{bitcoin}. This means that two parties can execute a secure financial transaction without relying on, e.g. a bank, to verify each transaction.

This is achieved via a distributed ledger system. The stored information is distributed across many nodes, which may be located anywhere in the world. Each transaction is transparent and secure, even without the parties' knowledge of each other. Transparency means that each transaction is immutably stored within the blocks and visible to anyone. Security is achieved via several measurements. Each block is hashed, meaning that tampering with data within a block leads to the entire block's data changing. The decentralized structure of the blockchain also means that each node has a copy of the blockchain, which makes it difficult to tamper with. \cite{blockchain}

Cryptocurrencies and NFTs are based on blockchain. 


%
% Section: Smart Contracts
%
\section{Smart Contracts}
\label{sec:background:smartcontracts}
Although Bitcoin does not natively support Smart Contracts, other blockchains such as Ethereum do. Smart Contracts are way to execute contracts between buyers and sellers, also without the need of a third party intervening. Once specific conditions of a contract are met, the underlying functions are automatically executed. \cite{smartContracts}

The purpose of Smart Contracts in regard to NFTs is to ensure their uniqueness and specify the terms of agreement. An NFT's Smart Contract might specify that the NFT will be transferred from one party to the other if one party pays the other the agreed upon amount.


%
% Section: Non-Fungible Tokens
%
\section{Non-Fungible Tokens}
\label{sec:background:nfts}


%
% Section: Wallets
%
\section{Cryptocurrency Wallets}
\label{sec:background:wallets}


%
% Section: Cookies
%
\section{Cookies}
\label{sec:background:cookies}



