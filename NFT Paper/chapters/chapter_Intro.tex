\chapter{Introduction}
\label{ch:intro}

Non-Fungible Tokens (NFTs) have gained popularity over the years with the rise of cryptocurrencies \cite{bayc}. Although it is unclear in which direction the future of the internet is heading, it can be speculated that the next iteration of the web, named Web3, will be focused on decentralized technologies such as blockchain. While the current internet is based on user-generated content, Web3 will be based on decentralized apps (dApps), and will run on peer-to-peer networks instead of central servers.

%
% Section: Motivation
%
\section{Motivation}
\label{sec:intro:motivation}
With NFTs being brought to light by recent trends in blockchain and cryptocurrencies such as Bitcoin, it is a very current and hot topic. NFTs have commonly been traded as a financial investment in their early years. The most prominent example being the Bored Apes Yacht Club, where some of the first minted NFTs sold for a total over \$24 Million in 2021 \cite{bayc}. This and many other examples go to show that there is a general interest in this up and coming technology.

Beyond financial investments, NFTs may also see potential use cases in areas of tracking user data. This is possible by the very nature of blockchain technology.


%
% Section: Overview
%
\section{Goals and Overview}
\label{sec:intro:overview}
The goal of this paper is to analyze to what degree and in what manner NFTs can be utilized to track user data in the web, similar to how cookies are currently used to track users across websites.

This will be done by giving background information on cookies and their privacy issues in chapter \ref{ch:bg_cookies}. Afterwards, the problem statement will be elaborated on in chapter \ref{ch:problem}. This will allow for an introduction of the required technical concepts of Web3 in chapter \ref{ch:bg_tech}. Chapter \ref{ch:SOTA} will then give an overview of current literature and market products in the areas of cookies and NFTs. Chapter \ref{ch:methodology} will then take a look at how NFTs can be leveraged in order to gain private information about a user and make content recommendations, similar to how cookies are able to do so. The results will then be discussed in chapter \ref{ch:results}. This paper will be concluded in chapter \ref{ch:conclusion} with a conclusion and a look into possible future research directions.